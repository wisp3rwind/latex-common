% TODO: check the physics package. On first impression, there are many commands
% I don't like in the very form they're defined in this package, but it might
% be a good idea to reimplement sane version of them here.

\usepackage{xparse}

\ExplSyntaxOn

\cs_new:Npn \__wisper_keys_define_phys:n #1 { \keys_define:nn { wisper/phys } {#1} }
\cs_new:Npn \__wisper_keys_set_phys:n #1 { \keys_set:nn { wisper/phys } {#1} }
\cs_new_eq:NN \PhysStyle \__wisper_keys_set_phys:n

\ExplSyntaxOff

% Don't rewrite things that already exist [[[1
% Use \verb|siunitx| to properly format numerical quantities.
\usepackage{siunitx}
\sisetup{%
  %locale=DE,
  separate-uncertainty = true,
  exponent-product = \cdot,
  range-units = brackets,
}

% Typeset istope symbols
\usepackage{isotope}
% silence compatibility warning by specifying a version.
\usepackage[version=4]{mhchem}
\providecommand\isomerslash{\textnormal{ / }\allowbreak}
\providecommand\isomer[1]{\cesplit{{\/}{\c{isomerslash}}}{#1}}

% Dirac notation
% This re-implements much of braket.sty because the latter has certain issues:
% - no support for mathtools-style size modifiers \braket[\big]{...}
% - interferes with my definition of \Set in common_math.tex
% - latex2e, therefore unreadable without fixing whitespace first
\ExplSyntaxOn

\cs_new:Npn \__wisper_size_left:nn #1#2
  {
    \IfBooleanTF {#1}
      { \left }
      {
        \__wisper_tl_if_novalue_or_blank:nF {#2} { \use:c { \cs_to_str:N #2 l } }
      }
  }
\cs_new:Npn \__wisper_size_mid:nn #1#2
  {
    \IfBooleanTF {#1}
      { \middle }
      {
        \__wisper_tl_if_novalue_or_blank:nF {#2} { #2 }
      }
  }
\cs_new:Npn \__wisper_size_right:nn #1#2
  {
    \IfBooleanTF {#1}
      { \right }
      {
        \__wisper_tl_if_novalue_or_blank:nF {#2} { \use:c { \cs_to_str:N #2 r } }
      }
  }
\DeclareDocumentCommand \ket {s o m}
  {
    \mathinner
      {
        \__wisper_size_mid:nn {#1} {#2} \vert
        #3
        \__wisper_size_right:nn {#1} {#2} \rangle
      }
  }
\DeclareDocumentCommand \bra {s o m}
  {
    \mathinner
      {
        \__wisper_size_mid:nn {#1} {#2} \langle
        #3
        \__wisper_size_right:nn {#1} {#2} \vert
      }
  }
\DeclareDocumentCommand \braket {s o m}
  {
    \group_begin:
      \char_set_mathcode:nn { `\| } { 32768 }  % active
      \cs_set:Npn \__wisper_mid_vert:
        {
          \, \__wisper_size_mid:nn {#1} {#2} \vert \,
        }
      \char_set_active_eq:NN | \__wisper_mid_vert: 
      \mathinner
        {
          \__wisper_size_left:nn {#1} {#2} \langle
          #3
          \__wisper_size_right:nn {#1} {#2} \rangle
        }
    \group_end:
  }
\DeclareDocumentCommand \ketbra {s o m m}
  {
    \mathinner
      {
        \__wisper_size_left:nn {#1} {#2} \vert
        #3
        \__wisper_size_mid:nn {#1} {#2} \rangle
        %\!
        \__wisper_size_mid:nn {#1} {#2} \langle
        #4
        \__wisper_size_right:nn {#1} {#2} \vert
      }
  }

% cf. https://tex.stackexchange.com/questions/214347/how-to-implement-a-macro-for-normal-ordering-of-operators
\NewDocumentCommand \NormalOrdering { m }
  {
    \vcentcolon \, \mathrel{#1} \, \vcentcolon
  }

\ExplSyntaxOff


% Electrodynamics [[[1
\newcommand{\epz}{\epsilon_0}
\newcommand{\muz}{\mu_0}
\newcommand{\epr}{\epsilon_r}
\newcommand{\mur}{\mu_r}

% Lagrange and Hamilton formalism [[[1
\newcommand{\lagr}{\mathcal{L}}
\newcommand{\hmlt}{\mathcal{H}}


% general quantum theory [[[1
\ExplSyntaxOn
\DeclarePairedDelimiterX \Comm[2] \lbrack \rbrack
  { \__wisper_argument_or_empty:n {#1}, \__wisper_argument_or_empty:n {#2} }
\ExplSyntaxOff

\newcommand{\paulimx}{\begin{pmatrix}0 & 1 \\ 1 & 0\end{pmatrix}}
\newcommand{\paulimy}{\begin{pmatrix}-i & 0 \\ 0 & i\end{pmatrix}}
\newcommand{\paulimz}{\begin{pmatrix}1 & 0 \\ 0 & -1\end{pmatrix}}

% vim: ts=4 sw=4 et fdm=marker fmr=[[[,]]]:
