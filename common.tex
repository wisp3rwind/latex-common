% This file should only contain things that can go into any kind of document.

% Use pgfkeys for everything [[[1
% Early packages loads required for handling the configuration.
\RequirePackage{pgfkeys}
\RequirePackage{xparse}
% TODO: ifs for: luatex, xelatex, beamer

% Key-value interface: convenient access to the namespace
\ExplSyntaxOn

\cs_new:Npn \__wisper_keys:n #1 { \pgfkeys{/wisper/.cd,#1} }
\cs_new:Npn \__wisper_doc_style:n #1 { \pgfkeys{/wisper/document/.cd,#1} }

\cs_new_eq:NN \DocumentStyle \__wisper_keys:n

% Document class [[[1
% Conditionals and switches [[[2
% should accept article, book, presentation, letter
\__wisper_doc_style:n {type/.initial=article}
\prg_new_conditional:Npnn \__wisper_if_document_type:n #1
  { T, F, TF}
  { 
    \__wisper_doc_style:n {type/.get=\tempa}
    \str_if_eq:nVTF {#1} \tempa
      { \prg_return_true: }
      { \prg_return_false: }
  }

\cs_new_eq:NN \IfDocTypeT \__wisper_if_document_type:nT
\cs_new_eq:NN \IfDocTypeF \__wisper_if_document_type:nF
\cs_new_eq:NN \IfDocTypeTF \__wisper_if_document_type:nTF

% TODO: automate the following definitions
\cs_new:Npn \IfPresentationT #1 { \__wisper_if_document_type:nT {presentation} {#1} }
\cs_new:Npn \IfPresentationF #1 { \__wisper_if_document_type:nF {presentation} {#1} }
\cs_new:Npn \IfPresentationTF #1#2 { \__wisper_if_document_type:nTF {presentation} {#1} {#2} }

\cs_new:Npn \IfLetterT #1 { \__wisper_if_document_type:nT {letter} {#1} }
\cs_new:Npn \IfLetterF #1 { \__wisper_if_document_type:nF {letter} {#1} }
\cs_new:Npn \IfLetterTF #1#2 { \__wisper_if_document_type:nTF {letter} {#1} {#2} }

\cs_new:Npn \IfArticleT #1 { \__wisper_if_document_type:nT {article} {#1} }
\cs_new:Npn \IfArticleF #1 { \__wisper_if_document_type:nF {article} {#1} }
\cs_new:Npn \IfArticleTF #1#2 { \__wisper_if_document_type:nTF {article} {#1} {#2} }

\cs_new:Npn \IfBookT #1 { \__wisper_if_document_type:nT {book} {#1} }
\cs_new:Npn \IfBookF #1 { \__wisper_if_document_type:nF {book} {#1} }
\cs_new:Npn \IfBookTF #1#2 { \__wisper_if_document_type:nTF {book} {#1} {#2} }

\cs_new:Npn \SwitchDocType #1#2
  { 
    \__wisper_doc_style:n {type/.get=\tempa}
    \str_case:onF \tempa
    { #1 }  % should contain lines of {type}{code} pairs
    { #2 }
  }

\ExplSyntaxOff

% Load the class [[[2
\SwitchDocType
  {
    {article}{\documentclass[a4paper]{article}}
    {presentation}{\documentclass{beamer}}
    {letter}{\documentclass{scrlttr2}}
    {book}{\documentclass{book}}
  }
  {
    % Invalid document type
    % TODO: use the .is choice handler to prevent this in the first place
  }

% Early Package loads [[[1

\usepackage[english]{babel}
\usepackage[T1]{fontenc}

% Font handling [[[1
\ExplSyntaxOn

\__wisper_doc_style:n
  {
    font/preset/.cd,
    .is~choice,

    lmodern/.style={
      internal/load~font/.code={
      }
    },

    fira/.style={
      internal/load~font/.code={
        % Mozilla Fira Sans, requires XeLaTeX or LuaLaTex TODO: raise error
        \usepackage[sfdefault]{FiraSans}
        \usepackage{FiraMono}
      }
    },

    garamond/.style={
      internal/load~font/.code={
        % URW-Garamond: A free garamond font, supposedly good for books but to fragile
        %   for presentations. Math fonts provided by the mathdesign project
        \usepackage[urw-garamond]{mathdesign}
        % FIXME: maybe drop this? PagellaX is better anyway. At least give a
        % warning.
        % TODO: matching sans font
      }
    },

    palatino/.style={
      internal/load~font/.code={
        \usepackage[osf]{mathpazo}
      }
    },

    TexGyre/.style={
      internal/load~font/.code={
        \usepackage{lmodern}
      }
    },

  }

\__wisper_doc_style:n {font/preset=TexGyre}
\__wisper_doc_style:n {internal/load~font}

\ExplSyntaxOff

% Load packages [[[1
% This file should only contain things that can go into any kind of document.

%\usepackage[german]{babel}

\newif\ifloadcode\loadcodetrue
\newif\ifloadmath\loadmathtrue
\newif\ifloadphys\loadphystrue

\usepackage[utf8]{inputenc}
\usepackage[T1]{fontenc}


% load minted before csquotes to suppress a warning due to a common package
% they load
\ifloadcode
  \usepackage{minted}
\fi

\usepackage{csquotes}  % context-sensitive quotes

\usepackage{microtype}


% Misc [[[1
\IfPresentationF{\usepackage[inline]{enumitem}}  % doesn't play well with overlay specifications
\usepackage{float}  %
%\usepackage{xparse}
\usepackage{calc}
%\usepackage{fancyhdr}  %
\usepackage{ragged2e}


% Colors [[[1
\usepackage{xcolor}


% Hyperlinks [[[1
\IfPresentationF{\usepackage[hyphens]{url}}  % loaded early by hyperref which is loaded way to early by beamer...
%\usepackage{hyperref}


% Better tables [[[1
\usepackage{rotating}
\usepackage{booktabs}
\usepackage{array}
%\usepackage{multirow}
%\usepackage{tabu}


% Graphics [[[1
\usepackage{graphicx}  % advanced key=value arguments for \includegraphics
\usepackage{caption}
\usepackage{subcaption}


% Bibliography [[[1
% many of these options can only be given in \usepackage and not later, unfortunately..
\IfPresentationTF
    {\usepackage[hyperref=true,
                 backref=true,
                 style=alphabetic,
                 citereset=section,
                 url=false,
                 isbn=false,%  DOI only
                 maxcitenames=3,
                 maxbibnames=3]{biblatex}}
    {\usepackage[hyperref=true,
                 backref=true,
                 %bibstyle=authoryear,
                 bibstyle=numeric,
                 %bibstyle=alphabetic,
                 citestyle=custom-numeric-comp,
                 %citestyle=numeric-comp,
                 %citestyle=custom-alphabetic,
                 citereset=section,
                 url=false,
                 isbn=false,%  DOI only
                 maxcitenames=3,
                 maxbibnames=3]{biblatex}}


% New packages to investigate [[[1
%\usepackage{float}
%\usepackage{color}
%\usepackage{subfigure}
%\usepackage{geometry}
%\usepackage{pdfpages}
%\usepackage{cite}  % better support for numeric citations
%\usepackage{longtable}  % wrap tables at page boundaries, succeeds supertabular
%\usepackage{supertabular}  % wrap tables at page boundaries
%\usepackage{wrapfig}
%\usepackage{footmisc}
%\usepackage{epstopdf}
%\usepackage{textcomp}
%\usepackage{marginnote}

% vim: ts=4 sw=4 et fdm=marker fmr=[[[,]]]:

% Packages that need to be loaded last [[[1
\usepackage{hyperref}
%\usepackage[inner]{showlabels}


% Configure packages, define commands [[[1
% Colors [[[2
% define a few highlighting colors, from https://personal.sron.nl/~pault/
\definecolor{marklightblue}{HTML}{BBCCEE}
\definecolor{marklightteal}{HTML}{CCEEFF}
\definecolor{marklightgreen}{HTML}{CCDDAA}
\definecolor{marklightsand}{HTML}{EEEEBB}
\definecolor{marklightred}{HTML}{FFCCCC}
\definecolor{markdarkblue}{HTML}{222255}
\definecolor{markdarkteal}{HTML}{225555}
\definecolor{markdarkgreen}{HTML}{225522}
\definecolor{markdarksand}{HTML}{666633}
\definecolor{markdarkred}{HTML}{663333}


% Tables [[[2
% https://tex.stackexchange.com/a/2442/111880
% https://tex.stackexchange.com/a/12712/111880
\newcolumntype{L}[1]{>{\raggedright\let\newline\\\arraybackslash\hspace{0pt}}m{#1}}
\newcolumntype{C}[1]{>{\centering\let\newline\\\arraybackslash\hspace{0pt}}m{#1}}
\newcolumntype{R}[1]{>{\raggedleft\let\newline\\\arraybackslash\hspace{0pt}}m{#1}}


% Hyperlinks [[[2
\hypersetup{%
    colorlinks=true,
    linkcolor=black,  % for internal links, color is just too much
    citecolor=markdarkteal,
    filecolor=markdarkgreen,
    urlcolor=markdarkblue,
    bookmarksopen=true,
    bookmarksopenlevel=3
}


% Bibliography [[[2
% From http://www.khirevich.com/latex/footnote_citation/
% new command \sjcitep prints footnote citation above punctuation
\newlength{\spc} % declare a variable to save spacing value
% TODO: ensure that this works when there are non-cite footnotes on the same page
\NewDocumentCommand{\citep}{D<>{} o o m}{% new command with two arguments: optional (#1) and mandatory (#2)
        \settowidth{\spc}{#1}% set value of \spc variable to the width of #1 argument
        \addtolength{\spc}{-1.8\spc}% subtract from \spc about two (1.8) of its values making its magnitude negative
        #1% print the optional argument
        \hspace*{\spc}% print an additional negative spacing stored in \spc after #1
        \IfNoValueTF{#2}
            {\supershortnotecite{#4}}
            {\IfNoValueTF{#3}
                {\supershortnotecite[#2]{#4}}
                {\supershortnotecite[#2][#3]{#4}}}
    }% print (cite) the mandatory argument


% New Commands [[[2
% unnumbered section that shows up in the TOC
\newcommand{\tocsection}[1]{\section*{#1}\addcontentsline{toc}{section}{#1}}
% non-numbered subsection that works properly with hyperref
\newcommand{\nonumchapter}[1]{\phantomsection\addcontentsline{toc}{chapter}{#1}}
\newcommand{\nonumsection}[1]{\phantomsection\addcontentsline{toc}{section}{#1}}
\newcommand{\nonumsubsection}[1]{\phantomsection\addcontentsline{toc}{subsection}{#1}}
\newcommand{\nonumsubsubsection}[1]{\phantomsection\addcontentsline{toc}{subsubsection}{#1}}

% Draft mode [[[2
%\renewcommand{\showlabelfont}{\tiny\ttfamily}
%\showlabels{cite}
%\showlabels{ref}
%\showlabels{begin}

% vim: ts=2 sw=2 et fdm=marker fmr=[[[,]]]:
