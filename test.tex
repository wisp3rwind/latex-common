% This file should only contain things that can go into any kind of document.

% Colors [[[1
% define a few highlighting colors, from https://personal.sron.nl/~pault/
\definecolor{marklightblue}{HTML}{BBCCEE}
\definecolor{marklightteal}{HTML}{CCEEFF}
\definecolor{marklightgreen}{HTML}{CCDDAA}
\definecolor{marklightsand}{HTML}{EEEEBB}
\definecolor{marklightred}{HTML}{FFCCCC}
\definecolor{markdarkblue}{HTML}{222255}
\definecolor{markdarkteal}{HTML}{225555}
\definecolor{markdarkgreen}{HTML}{225522}
\definecolor{markdarksand}{HTML}{666633}
\definecolor{markdarkred}{HTML}{663333}


% Tables [[[1
% https://tex.stackexchange.com/a/2442/111880
% https://tex.stackexchange.com/a/12712/111880
\newcolumntype{L}[1]{>{\raggedright\let\newline\\\arraybackslash\hspace{0pt}}m{#1}}
\newcolumntype{C}[1]{>{\centering\let\newline\\\arraybackslash\hspace{0pt}}m{#1}}
\newcolumntype{R}[1]{>{\raggedleft\let\newline\\\arraybackslash\hspace{0pt}}m{#1}}


% Hyperlinks [[[1
\hypersetup{%
    colorlinks=true,
    linkcolor=black,  % for internal links, color is just too much
    citecolor=markdarkteal,
    filecolor=markdarkgreen,
    urlcolor=markdarkblue,
    bookmarksopen=true,
    bookmarksopenlevel=3
}


% Bibliography [[[1


% New Commands [[[1
% unnumbered section that shows up in the TOC
\newcommand{\tocsection}[1]{\section*{#1}\addcontentsline{toc}{section}{#1}}
% non-numbered subsection that works properly with hyperref
\newcommand{\nonumchapter}[1]{\phantomsection\addcontentsline{toc}{chapter}{#1}}
\newcommand{\nonumsection}[1]{\phantomsection\addcontentsline{toc}{section}{#1}}
\newcommand{\nonumsubsection}[1]{\phantomsection\addcontentsline{toc}{subsection}{#1}}
\newcommand{\nonumsubsubsection}[1]{\phantomsection\addcontentsline{toc}{subsubsection}{#1}}

% Draft mode [[[1
\renewcommand{\showlabelfont}{\tiny\ttfamily}
%\showlabels{cite}
%\showlabels{ref}
%\showlabels{begin}

% vim: ts=4 sw=4 et fdm=marker fmr=[[[,]]]:

\DocumentStyle{main_language=english}
\Preamble{scrbook, font-TexGyre, math, phys, minted-default, koma-misc}

\ExplSyntaxOn
\NewDocumentCommand \Example { O{displaymath} v }
  {
    \texttt{#2}
    \quad
    \begin{#1}
      \tl_rescan:nn {}{#2}
    \end{#1}
  }
\ExplSyntaxOff

\usepackage{blindtext}
\usepackage{lipsum}

\title{\LaTeXe\ cheat sheet}
\author{Someone}
\date{}

\newif\iftesting
\testingtrue
%\testingfalse


\begin{document}

\maketitle

\section{Math}
\lipsum

\subsection{Proofs}

\begin{itemize}
	\item \verb|\qed|
	\item \verb|\begin{proof} \qedhere \end{proof}|
\end{itemize}

\subsection{Vector Calculus}
Bold vectors usually look better, \verb|\vc| is a shorthand:
\begin{table}[H]
  \centering
  \begin{tabular}{ll}
    \toprule
    \verb|\vc{a}| & $\vc{a}$ \\
    \verb|\Grad{f}| & $\Grad{f}$ \\
    \verb|\Div{\vc A}| & $\Div{\vc A}$ \\
    \verb|\Curl{\vc A}| & $\Curl{\vc A}$ \\
    \verb|\Laplacian{\vc{A}^2}| & $\Laplacian{\vc{A}^2}$ \\
    \bottomrule
  \end{tabular}
\end{table}


\NewDocumentCommand \vectorshowcase {} {%
  $\,\vec{a}\,\vec{A}\dvec{a}\ddvec{A}$\\
  $\vec{a}\,\vec{b}\,\vec{c}\,\vec{d}\,\vec{e}\,\vec{f}\,\vec{g}\,\vec{h}\,\vec{\imath}\,\vec{\jmath}\,\vec{k}\,\vec{l}\,\vec{m}\,\vec{n}\,\vec{o}\,\vec{p}\,\vec{q}\,\vec{r}\,\vec{s}\,\vec{t}\,\vec{u}\,\vec{v}\,\vec{w}\,\vec{x}\,\vec{y}\,\vec{z}$\\
  $\vec{A}\,\vec{B}\,\vec{C}\,\vec{D}\,\vec{E}\,\vec{F}\,\vec{G}\,\vec{H}\,\vec{I}\,\vec{J}\,\vec{K}\,\vec{L}\,\vec{M}\,\vec{N}\,\vec{O}\,\vec{P}\,\vec{Q}\,\vec{R}\,\vec{S}\,\vec{T}\,\vec{U}\,\vec{V}\,\vec{W}\,\vec{X}\,\vec{Y}\,\vec{Z}$
}

\MathStyle{vecdisplay=std}
\vectorshowcase

\MathStyle{vecdisplay=xvec}
\vectorshowcase

\MathStyle{vecdisplay=bm}
\vectorshowcase


\subsection{Derivatives}

\paragraph{Total, partial and functional derivatives}
\begin{itemize}
  \item \verb|\tderiv{f}{x}| \begin{displaymath} \tderiv{f}{x} \end{displaymath}
  \item \verb|\pderiv{f}{x}| \begin{displaymath} \pderiv{f}{x} \end{displaymath}
  \item \verb|\fderiv{A(\vc r)}{(\del_i \del_j A(\vc r))}| \begin{displaymath} \fderiv{A(\vc r)}{(\del_i \del_j A(\vec{r}))} \end{displaymath}
\end{itemize}

\paragraph{Advanced variable/exponent specification}
\begin{itemize}
  \item \verb|\pderiv{f}{x^2,y,z^2}| \begin{displaymath} \pderiv{f}{x^2,y,z^2} \end{displaymath}
  \item \verb|\pderiv{f}[n+m+1]{x^n,y,z^m}| \begin{displaymath} \pderiv{f}[n+m+1]{x^n,y,z^m} \end{displaymath}
  \item \verb|\pderiv{f}{x^n,y,z^m}| \begin{displaymath} \pderiv{f}[n+m+1]{x^n,y,z^m} \end{displaymath}
\end{itemize}

\paragraph{Shorthands for wrapping in parenthesis or brackets}
\begin{itemize}
  \item \verb|\ppderiv{f}{x^2,y,z^2}| \begin{displaymath} \ppderiv{f}{x^2,y,z^2} \end{displaymath}
  \item \verb|\bpderiv{f}[n+m+1]{x^n,y,z^m}| \begin{displaymath} \bpderiv{f}[n+m+1]{x^n,y,z^m} \end{displaymath}
\end{itemize}

\subsection{Integrals and differential forms}
\verb|\dif|, \verb|Dif| \verb|\del| and \verb|var| automatically provide
correct spacing (or more precisely, wrap the argument in \verb|\mathinner| and let
\TeX do the job).
In case they don't, use the starred version to disable the spacing. Also, if the
argument is empty, no spacing will be applied.
\begin{itemize}
  \item \verb|\dif{x} \dif^3 x \dif{x^3} \dif{{x^3}} \dif^3(\cos x) \dif({\cos x}^3) \dif({\cos x^3})|
    \begin{displaymath}
      \dif{x} \quad
      \dif^3 x \quad
      \dif{x^3} \quad
      \dif{{x^3}} \quad
      \dif^3(\cos x) \quad
      \dif({\cos x}^3) \quad
      \dif({\cos x^3})
    \end{displaymath}
  \item \verb|\dif \Dif \del \var|
    \begin{displaymath} \dif \quad \Dif \quad \del \quad \var \end{displaymath}
  \item \verb|\int\dif{x} x^2 \int x^2 \dif{x}|
    \begin{displaymath} \int\dif{x} x^2 \qquad \int x^2 \dif{x} \end{displaymath}
  \item
    \verb|\MathStyle{integrate/displayfunc=outset} \Int{x^2}{x} \qquad| \\
    \verb|\MathStyle{integrate/displayfunc=inset} \Int{x^2}{x}|
    \begin{displaymath}
      \MathStyle{integrate/displayfunc=outset} \Int{x^2}{x} \qquad
      \MathStyle{integrate/displayfunc=inset} \Int{x^2}{x}
    \end{displaymath}
  \item \Example|\Int[outset]{x^2}{x} \qquad \Int[inset]{x^2}{x}|
  \item Let's reset the style for the rest of this document:
    \Example|\MathStyle{integrate/displayfunc=outset}|
  \item \Example|\var S = \Int{t}{\Int{\vc r^3}{\vc{r}} \lagr(t, \vc{r}, \dvc{r})}|
  \item Limits can easily be specified: \\
    \Example|\Int[inset]{x^3,\R^3}{f(x)} \qquad \Int[outset]{x^3,\R^3}{f(x)}|
  \item and the long form of the differential is also supported: \\
    \Example|\Int[outset](\cos t,-\infty,\infty){f(t)} \\ \qquad \Int[inset](\cos t,-\infty,\infty){f(t)}|
  \item \verb|\dif{f} = \pderiv{f}{x} \dif{x} + \pderiv{f}{y} \dif y| (note the incorrect spacing of $\dif y$ (Why is that so? Currently, I don't understand this spacing issue myself\ldots)) \begin{displaymath} \dif{f} = \pderiv{f}{x} \dif{x} + \pderiv{f}{y} \dif y \end{displaymath}
\end{itemize}

\subsection{Symbols}

\subsection{Big operators}
These all support the same syntax
\begin{itemize}
  \item \Example|\Prod{a_i}|
  \item \Example|\Prod{a_i}|
  \item \Example|\Prod[i, 1, N]{a_i}|
  \item \Example|\Sum[i, 1, N]{a_i}|
  \item \Example|\Sum[i \in I]{a_i}|
  \item \Example|{(ABC)}_{ij} = \Prod[{k,l}, 1, N]{a_{ik} b_{kl} c_{lj}}|
  \item \Example|X = \Coprod[j \in J]{X_j}|
  \item \Example|X = \TensorProd[j \in J]{X_j}|
  \item \Example|X = \Union[j \in J]{X_j}|
  \item \Example|X = \Intersection[j \in J]{X_j}|
\end{itemize}

\subsubsection{Sets}

\begin{itemize}
  \item \Example|A \setminus B|
  \item \Example|\emptyset|
  \item \Example|A \subset B|
  \item \Example|A \supset B|
  \item \Example|A \subseteq B|
  \item \Example|A \intersection B|
  \item \Example|A \union B|
  \item \Example|A \in B|
\end{itemize}


\subsubsection{Logic}

\begin{itemize}
  \item \Example|\exists x \in X|
  \item \Example|A \iff B|
  \item \Example|\forall x \in X|
  \item \Example|A \implies B|
  \item \Example|A \implies \neg B|
\end{itemize}

\subsubsection{Arrows}
%math arrows


\subsubsection{Limits etc.}

\begin{itemize}
	\item \verb|\inf| $\inf$
	\item \verb|\sup| $\sup$
	\item \verb|n \to \infty| $n \to \infty$
%\sup
%\to
\end{itemize}


\subsection{Stacked relations and braces}
%\stackrel
%\underbrace
%\overset


\subsection{Cases}
\begin{itemize}
  \item \Example|h(x) = \begin{cases}f(x) & \q*{if} x < 0 \\ g(x) & \q*{if} x \ge 0 \end{cases}|
\end{itemize}


\subsection{Math fonts}


\subsection{Text in math}
% \text isn't plain latex but an ams addition
% better than \textrm as it gets the font right
% better than \textnormal as it gets the size right (e.g. for subscripts)(Is this true?)
% subscripts that aren't variables should be upright and correctly spaced (i.e. not \mathrm): which do get this right?
% ams \textup ?
%\text
%\textrm
% See http://stefaanlippens.net/textnormal
%\textnormal % prefer over \textrm, as the latter always uses a roman font, which in most cases, but not always is correct

\subsection{math environments incl. IEEE}


\section{Labels and references}

\section{Physics}

\subsection{General symbols}

\begin{description}
	\item[Lagrangian] \verb|\lagr| \quad $\lagr$
	\item[Hamiltonian] \verb|\hmlt| \quad $\hmlt$
\end{description}


\subsection{Quantum mechanics: general}

\begin{description}
  \item[Commutator] \verb|\Comm{H}{\rho}| \\
    \begin{displaymath}
      \Comm{H}{\rho}
    \end{displaymath}
  \item[Pauli matrices] \verb|\paulimx \paulimy \paulimz| \\
    \begin{displaymath}
      \paulimx \quad \paulimy \quad \paulimz
    \end{displaymath}
\end{description}


\subsection{Number formatting}
Use \verb|siunitx|, see there.

\subsection{Isotopes}
Use the \verb|isotope| package.

\subsection{Dirac Notation}
From the \verb|braket| package.
\begin{table}[H]
  \centering
  \begin{tabular}{rll}
    \toprule
    bra & \verb|\bra{\varphi}| & \quad $\bra{\varphi}$ \\
    ket & \verb|\ket{\psi}| & \quad $\ket{\psi}$ \\
    scalar product & \verb+\braket{\varphi | psi}+ & \quad $\braket{\varphi | \psi}$ \\
    matrix element & \verb+\braket{\varphi | \hmlt | \psi}+ & \quad $\braket{\varphi | \hmlt | \psi}$ \\
    expectation value & \verb+\braket{\hmlt}+ & \quad $\braket{\hmlt}$ \\
    \bottomrule
  \end{tabular}
\end{table}

\section{Tables}

% TODO: booktabs

\begin{sidewaystable}
	\centering
	\begin{tabular}{rcl}
		%\hline
		%\multirow{n}
	\end{tabular}
\end{sidewaystable}

\section{Boxes}

\NewFancyBox { bluebox } { bar-colour=q-vibrant-blue, bg-colour=q-vibrant-blue!8 }
\NewFancyBox { cyanbox } { bar-colour=q-vibrant-cyan, bg-colour=q-vibrant-cyan!8 }
\NewFancyBox { tealbox } { bar-colour=q-vibrant-teal, bg-colour=q-vibrant-teal!8 }
\NewFancyBox { orangebox } { bar-colour=q-vibrant-orange, bg-colour=q-vibrant-orange!8 }
\NewFancyBox { redbox } { bar-colour=q-vibrant-red, bg-colour=q-vibrant-red!8 }
\NewFancyBox { magentabox } { bar-colour=q-vibrant-magenta, bg-colour=q-vibrant-magenta!8 }
\NewFancyBox { greybox } { bar-colour=q-vibrant-grey, bg-colour=q-vibrant-grey!8 }
\begin{bluebox} Blue box \end{bluebox}
\begin{cyanbox} Cyan box \end{cyanbox}
\begin{tealbox} Teal box \end{tealbox}
\begin{orangebox} Orange box \end{orangebox}
\begin{redbox} Red box \end{redbox}
\begin{magentabox} Magenta box \end{magentabox}
\begin{greybox} Grey box \end{greybox}

\section{Code}

\begin{itemize}
	\item \verb|\inputminted{file.py}|

  \item Keep all whitespace from the source file \begin{verbatim}
    \begin{minted}{python}
      i = False if a == "foo" else "bar"
      for k in range(10):
        print(k)
    \end{minted}
    \end{verbatim}
    \begin{minted}{python}
      i = False if a == "foo" else "bar"
      for k in range(10):
        print(k)
    \end{minted}

  \item Remove additional whitespace \begin{verbatim}
    \begin{minted}[autogobble]{python}
      i = False if a == "foo" else "bar"
      for k in range(10):
        print(k)
    \end{minted}
    \end{verbatim}
    \begin{minted}[autogobble]{python}
      i = False if a == "foo" else "bar"
      for k in range(10):
        print(k)
    \end{minted}

  \item \verb+\mint{python}|import this|+ \mint{python}|import this|

  \item \verb+\mintinline{python}|import this|+ \quad \mintinline{python}|import this|
\end{itemize}

\paragraph{Styles} Pygments comes with a number of built-in styles that can
be used with \verb|\usemintedstyle[language]{style}|.
\begin{sidewaystable}
  \begin{tabular}{ll}
    \toprule
      default & \usemintedstyle{default} \mintinline{python}|a = [v for i, (k, v) in enumerate(d) if k in selection]  # Whatever.| \\
      emacs & \usemintedstyle{emacs}  \mintinline{python}|a = [v for i, (k, v) in enumerate(d) if k in selection]  # Whatever.| \\
      friendly & \usemintedstyle{friendly}  \mintinline{python}|a = [v for i, (k, v) in enumerate(d) if k in selection]  # Whatever.| \\
      colorful & \usemintedstyle{colorful}  \mintinline{python}|a = [v for i, (k, v) in enumerate(d) if k in selection]  # Whatever.| \\
      autumn & \usemintedstyle{autumn}  \mintinline{python}|a = [v for i, (k, v) in enumerate(d) if k in selection]  # Whatever.| \\
      murphy & \usemintedstyle{murphy}  \mintinline{python}|a = [v for i, (k, v) in enumerate(d) if k in selection]  # Whatever.| \\
      manni & \usemintedstyle{manni}  \mintinline{python}|a = [v for i, (k, v) in enumerate(d) if k in selection]  # Whatever.| \\
      monokai & \usemintedstyle{monokai}  \mintinline{python}|a = [v for i, (k, v) in enumerate(d) if k in selection]  # Whatever.| \\
      perldoc & \usemintedstyle{perldoc}  \mintinline{python}|a = [v for i, (k, v) in enumerate(d) if k in selection]  # Whatever.| \\
      pastie & \usemintedstyle{pastie}  \mintinline{python}|a = [v for i, (k, v) in enumerate(d) if k in selection]  # Whatever.| \\
      borland & \usemintedstyle{borland}  \mintinline{python}|a = [v for i, (k, v) in enumerate(d) if k in selection]  # Whatever.| \\
      trac & \usemintedstyle{trac}  \mintinline{python}|a = [v for i, (k, v) in enumerate(d) if k in selection]  # Whatever.| \\
      native & \usemintedstyle{native}  \mintinline{python}|a = [v for i, (k, v) in enumerate(d) if k in selection]  # Whatever.| \\
      fruity & \usemintedstyle{fruity}  \mintinline{python}|a = [v for i, (k, v) in enumerate(d) if k in selection]  # Whatever.| \\
      bw & \usemintedstyle{bw}  \mintinline{python}|a = [v for i, (k, v) in enumerate(d) if k in selection]  # Whatever.| \\
      vim & \usemintedstyle{vim}  \mintinline{python}|a = [v for i, (k, v) in enumerate(d) if k in selection]  # Whatever.| \\
      vs & \usemintedstyle{vs}  \mintinline{python}|a = [v for i, (k, v) in enumerate(d) if k in selection]  # Whatever.| \\
      tango & \usemintedstyle{tango}  \mintinline{python}|a = [v for i, (k, v) in enumerate(d) if k in selection]  # Whatever.| \\
      rrt & \usemintedstyle{rrt}  \mintinline{python}|a = [v for i, (k, v) in enumerate(d) if k in selection]  # Whatever.| \\
      xcode & \usemintedstyle{xcode}  \mintinline{python}|a = [v for i, (k, v) in enumerate(d) if k in selection]  # Whatever.| \\
      igor & \usemintedstyle{igor}  \mintinline{python}|a = [v for i, (k, v) in enumerate(d) if k in selection]  # Whatever.| \\
      paraiso-light & \usemintedstyle{paraiso-light}  \mintinline{python}|a = [v for i, (k, v) in enumerate(d) if k in selection]  # Whatever.| \\
      paraiso-dark & \usemintedstyle{paraiso-dark}  \mintinline{python}|a = [v for i, (k, v) in enumerate(d) if k in selection]  # Whatever.| \\
      lovelace & \usemintedstyle{lovelace}  \mintinline{python}|a = [v for i, (k, v) in enumerate(d) if k in selection]  # Whatever.| \\
      algol & \usemintedstyle{algol}  \mintinline{python}|a = [v for i, (k, v) in enumerate(d) if k in selection]  # Whatever.| \\
      algol\_nu & \usemintedstyle{algol_nu}  \mintinline{python}|a = [v for i, (k, v) in enumerate(d) if k in selection]  # Whatever.| \\
      arduino & \usemintedstyle{arduino}  \mintinline{python}|a = [v for i, (k, v) in enumerate(d) if k in selection]  # Whatever.| \\
      rainbow\_dash & \usemintedstyle{rainbow_dash}  \mintinline{python}|a = [v for i, (k, v) in enumerate(d) if k in selection]  # Whatever.| \\
      abap & \usemintedstyle{abap}  \mintinline{python}|a = [v for i, (k, v) in enumerate(d) if k in selection]  # Whatever.| \\
    \bottomrule
  \end{tabular}
\end{sidewaystable}

%\paragraph 
%\begin{shortpycode}
%  # ...
%\end{shortpycode}
%\begin{shortpycode*}
%  # ...
%\end{shortpycode*}
%\mintinline{python}{# ...}

\section{Scratch space}

\end{document}



%==============================================================================

%\clearpage

%\noindent

%\begin{flushleft}\end{flushleft}
%\begin{enumerate}\end{enumerate}
%\begin{enumerate}[label=...]\end{enumerate}
%\begin{itemize}\end{itemize} % including inline!
%\begin{description}\end{description}
%labels on images, tables, \item, math, paragraph
%\pagebreak[4]
%\usepackage{marginnote}
%%\renewcommand{\thefootnote}{\fnsymbol{footnote}}
%\renewcommand*{\marginfont}{\footnotesize}
%\usepackage{mchem}
%tikz, cf. particle presentation
%images
%tikz-feynman
