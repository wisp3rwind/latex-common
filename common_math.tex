% TODO: check other dependencies, i.e. xparse, ?, i.e. don't fail, if they're 
% already loaded
% TODO: add ensuremath everywhere where appropriate
\usepackage{xparse}
\usepackage{ifthen,calc}

\usepackage{amsmath,amstext,amssymb,amsthm,amsfonts}
\usepackage{mathtools}


% symbols
% TODO: switch according to language, i.e. Spanish is cte.
\newcommand{\const}{\mathrm{const}}


% sets
\newcommand{\N}{\ensuremath{\mathbb{N}}}
\newcommand{\Z}{\ensuremath{\mathbb{Z}}}
\newcommand{\R}{\ensuremath{\mathbb{R}}}
\newcommand{\Q}{\ensuremath{\mathbb{Q}}}

\newcommand{\nats}{\N} % natural numbers
\newcommand{\natsz}{\ensuremath{\nats_{\mathrm{0}}}} % natural numbers incl. zero

\newcommand{\Pset}[1]{\ensuremath{\mathcal{P}\left(#1\right)}}


% norms etc.
\DeclarePairedDelimiter\abs{\lvert}{\rvert}
\DeclarePairedDelimiter\Abs{\left\lvert}{\right\rvert}
\DeclarePairedDelimiter\norm{\lVert}{\rVert}
\DeclarePairedDelimiter\Norm{\left\lVert}{\right\rVert}


% ranges
% TODO: let these accept a variable number of initial values, e.g.
% \idxrange{1}[1][2][3][5]{n}
\newcommand{\idxrange}[3]{{#1}_{#2}\dots{#1}_{#3}}
\newcommand{\idxrangec}[3]{{#1}_{#2},\dots{#1}_{#3}}  % comma-separated

% complex numbers
\newcommand{\conj}[1]{#1^{\ast}}


% vectors etc.
\newcommand{\vecb}[1]{\mathbf{#1}}
\newcommand{\colvec}[1]{\begin{pmatrix}#1\end{pmatrix}}


% operators
\newcommand{\adj}[1]{#1^{\dagger}}
\newcommand{\op}[1]{\hat{#1}}

\newcommand{\vecnabla}{\vec{\nabla}}
\newcommand{\grad}{\nabla}
\newcommand{\diverg}{\vecnabla\cdot}
\newcommand{\rot}{\vecnabla\times}


% calculus
\newcommand{\dif}{\mathrm{d}}
\newcommand{\Dif}{\mathrm{D}}
\newcommand{\postdif}[1]{\, \dif #1}
\newcommand{\PostDif}[1]{\, \Dif #1}
\newcommand{\predif}[1]{\dif #1 \,}
\newcommand{\PreDif}[1]{\Dif #1 \,}

% Lebesgue formalism
\newcommand{\leb}{\ensuremath{\lambda}}

% functions
\newcommand{\func}[1]{\mathrm{#1}}
\newcommand{\expp}[1]{\func{e}^{#1}}
\newcommand{\expb}[1]{\exp\left(#1\right)}
\newcommand{\deltaf}[1]{\func{\delta} (#1)}
\newcommand{\asinh}{\func{asinh}}

% Derivatives

% TODO: consolidate \*deriv and \*derivn by making the order parameter optional
\newcommand{\totderiv}[2]{\deriv{\dif}{#1}{#2}}
\newcommand{\totderivn}[3]{\derivn{\dif}{#1}{#2}{#3}}
\NewDocumentCommand{\totderivx}{o m >{\SplitList{;}}m}{%
    \frac{\IfNoValueTF{#1}{\dif}{\dif^{#1}} #2}
         {\ProcessList{#3}{\totderivxadd}}
}
\newcommand{\totderivxadd}[1]{\dif{#1}}

\newcommand{\parderiv}[2]{\deriv{\partial}{#1}{#2}}
\newcommand{\parderivn}[3]{\derivn{\partial}{#1}{#2}{#3}}
\NewDocumentCommand{\parderivx}{o m >{\SplitList{;}}m}{%
    \frac{\IfNoValueTF{#1}{\partial}{\partial^{#1}} #2}
         {\ProcessList{#3}{\parderivxadd}}
}
\newcommand{\parderivxadd}[1]{\partial{#1}}

\newcommand{\deriv}[3]{%
    % #1 derivative sign
    % #2 function
    % #3 variables
    \frac{#1 #2}{#1 #3}
}
\newcommand{\derivn}[4]{%
    % #1 derivative sign
    % #2 order
    % #3 function
    % #4 variables
    \frac{#1^{#2} #3}{#1 #4^{#2}}
}


% equation numbering
%http://tex.stackexchange.com/questions/42726/align-but-show-one-equation-number-at-the-end
\newcommand\numberthis{\addtocounter{equation}{1}\tag{\theequation}}

% vim: ts=4 sw=4 et :
