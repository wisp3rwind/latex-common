% TODO: check other dependencies, i.e. xparse, ?, i.e. don't fail, if they're 
% already loaded
% TODO: add ensuremath everywhere where appropriate
\usepackage{xparse}
\usepackage{ifthen,calc}

\usepackage{amsmath,amstext,amssymb,amsthm,amsfonts}
\usepackage{mathtools}


% symbols
% TODO: switch according to language, i.e. Spanish is cte.
\newcommand{\const}{\mathrm{const}}


% sets
\newcommand{\N}{\ensuremath{\mathbb{N}}}
\newcommand{\Z}{\ensuremath{\mathbb{Z}}}
\newcommand{\R}{\ensuremath{\mathbb{R}}}
\newcommand{\Q}{\ensuremath{\mathbb{Q}}}

\newcommand{\nats}{\N} % natural numbers
\newcommand{\natsz}{\ensuremath{\nats_{\mathrm{0}}}} % natural numbers incl. zero

\newcommand{\Pset}[1]{\ensuremath{\mathcal{P}\left(#1\right)}}


% norms etc.
\DeclarePairedDelimiter\abs{\lvert}{\rvert}
\DeclarePairedDelimiter\norm{\lVert}{\rVert}


% ranges
% TODO: let these accept a variable number of initial values, e.g.
% \idxrange{1}[1][2][3][5]{n}
\newcommand{\idxrange}[3]{{#1}_{#2}\dots{#1}_{#3}}
\newcommand{\idxrangec}[3]{{#1}_{#2},\dots{#1}_{#3}}  % comma-separated

% complex numbers
\newcommand{\conj}[1]{#1^{\ast}}


% vectors etc.
\newcommand{\vecb}[1]{\mathbf{#1}}
\newcommand{\colvec}[1]{\begin{pmatrix}#1\end{pmatrix}}


% operators
\newcommand{\adj}[1]{#1^{\dagger}}
\newcommand{\op}[1]{\hat{#1}}

\newcommand{\vecnabla}{\vec{\nabla}}
\newcommand{\grad}{\nabla}
\newcommand{\diverg}{\vecnabla\cdot}
\newcommand{\rot}{\vecnabla\times}


% calculus
\newcommand{\dif}{\mathrm{d}}
\newcommand{\Dif}{\mathrm{D}}
\newcommand{\postdif}[1]{\, \dif #1}
\newcommand{\PostDif}[1]{\, \Dif #1}
\newcommand{\predif}[1]{\dif #1 \,}
\newcommand{\PreDif}[1]{\Dif #1 \,}

% Lebesgue formalism
\newcommand{\leb}{\ensuremath{\lambda}}

% functions
\newcommand{\func}[1]{\mathrm{#1}}
\newcommand{\expp}[1]{\func{e}^{#1}}
\newcommand{\expb}[1]{\exp\left(#1\right)}
\newcommand{\deltaf}[1]{\func{\delta} (#1)}
\newcommand{\asinh}{\func{asinh}}

% Derivatives

% TODO: consolidate \*deriv and \*derivn by making the order parameter optional
\newcommand{\totderiv}[2]{\deriv{\dif}{#1}{#2}}
\newcommand{\totderivn}[3]{\derivn{\dif}{#1}{#2}{#3}}
\NewDocumentCommand{\totderivx}{o m >{\SplitList{;}}m}{%
    \frac{\IfNoValueTF{#1}{\dif}{\dif^{#1}} #2}
         {\ProcessList{#3}{\totderivxadd}}
}
\newcommand{\totderivxadd}[1]{\dif{#1}}

\newcommand{\parderiv}[2]{\deriv{\partial}{#1}{#2}}
\newcommand{\parderivn}[3]{\derivn{\partial}{#1}{#2}{#3}}
\NewDocumentCommand{\parderivx}{o m >{\SplitList{;}}m}{%
    \frac{\IfNoValueTF{#1}{\partial}{\partial^{#1}} #2}
         {\ProcessList{#3}{\parderivxadd}}
}
\newcommand{\parderivxadd}[1]{\partial{#1}}

\newcommand{\deriv}[3]{%
    % #1 derivative sign
    % #2 function
    % #3 variables
    \frac{#1 #2}{#1 #3}
}
\newcommand{\derivn}[4]{%
    % #1 derivative sign
    % #2 order
    % #3 function
    % #4 variables
    \frac{#1^{#2} #3}{#1 #4^{#2}}
}


%\NewDocumentCommand{\parderivx}{o m m}{\derivx{\partial}[#1]{#2}{#3}}
\NewDocumentCommand{\derivx}{m o m >{\SplitList{;}}m}{%
    % #1 derivative sign
    % #2 order, optional
    % #3 function
    % #4 variables as a semicolon-separated list, e.g `x^2;y;z^3`
    \frac{\IfNoValueTF{#2}{#1}{#1^{#2}} #3}
         {\ProcessList{#4}{\derivxadd{#1}}}  % TODO: this doesn't work, fix
}
\newcommand{\derivxadd}[2]{#1 {#2}}

%------------------------------------------------------------

% TODO: automatically count order (command should be backwards-compatible to the above \totderiv so it can completely replace it)
\newcounter{derivordercounter}
% example: \totderiv{f}{x}{x_exp}{y}{y_exp}
% example: \totderiv{f}{x^2;y^3} -> more readable
% optional argument to overwrite the count:
%   \totderiv{f}[n+m]{x^n;y^m}
%   -> needs to suppress \addtocounter calls which would fail
%       -> maybe simply a starred version of the command?
% TODO:
% make \totderiv and \parderiv call a more general command by providing
% the dif sign as parameter
\NewDocumentCommand \derivnew {m m o >{\SplitList{;}}m} {%
    % #1 - deriv sign
    % #2 - function
    % #3 - exponent overwrite for non-numerical cases
    % list - semicolon-separated list of variables, exponents given as usual with ^
    \IfNoValueTF{#3}{%
        % no \addtocounter overwrite
        % determine order, no output yet
        {\ProcessList{#4}{\derivcountingsplitter}}
        \frac{%
            #1^{%
                \ifthenelse{\lessorequal{\derivordercounter}{1}}{%
                    % nothing, i.e. do not show exponent if it is 1
                }{%
                    \thederivordercounter
                }
            } 
            \IfNoValueTF{#2}{%
                % nothing
            }{%
                #2
            }
        }{%
            \ProcessList{#4}{\derivaddingsplitter{#1}}
        }
        \setcounter{\derivordercounter}{0}
    }{%  else
        \frac{%
            \dif^{#3} #2
        }{%
            \ProcessList{#4}{\derivaddingsplitter{#1}}
        }
    }
}

\NewDocumentCommand \derivcountingsplitter {>{\SplitArgument{2}{^}}m} {%
    \derivcount #1
}
\NewDocumentCommand \derivcount {m m} {%
    \addtocounter{\derivordercounter}{\IfNoValueTF{#2}{1}{#2}}
    %#1\IfNoValueTF{#2}{}{^#2}
}

\NewDocumentCommand \derivaddingsplitter {m >{\SplitArgument{2}{^}}m} {%
    % SplitArgument already wraps the parts in #2 in braces
    \derivadd{#1}#2
}
\NewDocumentCommand \derivadd {m m m} {%
    % TODO: wrap everything but the exponent in parentheses if "long" (?)
    % or better: if #2 is surround by (,),\left(,right) strip them and add
    % again before the derivative sign #1
    #1 #2\IfNoValueTF{#3}{}{^#3}
}

% equation numbering
%http://tex.stackexchange.com/questions/42726/align-but-show-one-equation-number-at-the-end
\newcommand\numberthis{\addtocounter{equation}{1}\tag{\theequation}}

% vim: ts=4 sw=4 et :
